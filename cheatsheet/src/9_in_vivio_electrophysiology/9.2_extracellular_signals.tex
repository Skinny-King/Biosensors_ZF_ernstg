\subsection{Extracellular Signals}

\subsubsection*{Biophysics of Extracellularly Recorded Spikes}
Action potentials create spatial voltage differences along the membrane, inducing extracellular currents that are detected by microelectrodes.

\textbf{Axonal spikes:}
Typically triphasic waveform $(+,-,+)$.

\textbf{Somatic spikes:}
Typically biphasic waveform $(-,+)$.

\textbf{Neuron type:}
Broad spikes are associated with excitatory neurons, narrow spikes with inhibitory neurons.

\subsubsection*{Extracellular vs. Intracellular Recordings}
\begin{tabular}{l | l | l}
 & Intracellular & Extracellular \\
\hline
Units & Single cell & One or more cells \\
Signal & Membrane potential & Local extracellular potential \\
Duration & Minutes--hours & Days--months \\
Amplitude & mV & 10--100 $\mu$V \\
Use & In vitro & In vivo
\end{tabular}

\subsubsection*{Local Field Potential}
The LFP reflects the summed low-frequency ($\sim$1--300 Hz) extracellular potentials generated by transmembrane currents from many neurons, including synaptic, dendritic, somatic, and glial activity.

\subsubsection*{Spike Detection}
\textbf{Pipeline:}
Raw signal
$\rightarrow$
band-pass filter (300--6000 Hz)
$\rightarrow$
noise estimate $\sigma=\mathrm{median}(|y|)/0.6745$
$\rightarrow$
threshold $T=-k\sigma$ ($k\approx4.5$)
$\rightarrow$
downward crossings
$\rightarrow$
refractory period
$\rightarrow$
aligned spike times.

\subsubsection*{Multiple Recording Sites}
Closely spaced multi-site probes ($<50\,\mu$m) record the same neuron on multiple channels with different amplitudes and waveforms.  
This improves spike sorting, spatial localization, and separation of nearby neurons.
