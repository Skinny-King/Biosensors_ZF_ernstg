\subsection*{Resting Potential and Ion Concentrations}

\begin{itemize}
    \item $[K^+]_{in} > [K^+]_{out}$ \hfill $[Na^+]_{in} < [Na^+]_{out}$
    \item $[K^+]_{in} \uparrow \;\Rightarrow\; V_{m} \downarrow$ \hfill $[K^+]_{out} \uparrow \;\Rightarrow\; V_{m} \uparrow$
    \item $[K^+]_{in} \approx [K^+]_{out} \;\Rightarrow\; V_{m} \approx 0$
    \item Effect of changes in $[Na^+]$ $<$ effect of changes in $[K^+]$
\end{itemize}

% ----------------

% ----------------
\begin{minipage}[t]{0.51\linewidth}
\textbf{Multiple-ion model:}
    \[
    I_{12} = G_{12} (\overbracket[0.4pt]{V_{12}}^{\text{$V_1 - V_2$}} - E_{12})
    \]
\end{minipage}
\begin{minipage}[t]{0.51\linewidth}
\vspace{0pt}
\tiny
    $I$ : membrane current $[A]$ \\
    $V$ : membrane potential $[V]$ \\
    $E$ : membraneNernst potential $[V]$ \\
\end{minipage}

\begin{minipage}[t]{0.65\linewidth}
    \[
    \boxed{
    V_{m}
    = \frac{G_{Na}V_{Na} + G_K V_K + G_o V_o}
    {G_{Na}+G_K+G_o}
    }
    \]

    \[
    \boxed{
    I_{m} = 0
    \;\Rightarrow\;
    V_{m}^0
    = \sum_n \frac{G_n}{\sum_k G_k}\,V_n
    }
    \]
\end{minipage}
\begin{minipage}[t]{0.5\linewidth}
\vspace{0pt}
\includegraphics[width=0.6\linewidth]{src/0_images/membr_in_parallel.png}
\end{minipage}
\includegraphics[width=0.4\linewidth]{src/0_images/comp_in_series.png}
\includegraphics[width=0.6\linewidth]{src/0_images/membr_in_series.png}


Rest vs Equilibrium potential:
