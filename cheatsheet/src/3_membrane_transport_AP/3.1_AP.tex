\subsection*{Action Potential (AP)}
\includegraphics[width=1\linewidth]{src/0_images/AP.png}

\textbf{The action potential consists of:}
\begin{enumerate}
	\item At rest, there is a steady influx of sodium ions and efflux of potassium ions
	\item An increase in the membrane potential due to the stimulus current
	\item A larger deflection of the transmembrane voltage caused by sodium ion influx
	\item A slower recovery to resting conditions caused by the efflux of potassium ions
	\item Depolarizing or hyperpolarizing afterpotentials may be observed
\end{enumerate}

\subsubsection*{Attributes of Action Potentials}

\textbf{Threshold:}
AP occurs only if threshold is reached; waveform is stimulus-independent.
Stronger stimuli shorten latency (more successive APs).\\

\textbf{Fiber diameter:}
Threshold \(\propto 1/\sqrt{d}\);
conduction velocity \(\propto \sqrt{d}\)
(thicker fibers → lower threshold, faster AP).\\

\textbf{Refractory periods:}
\emph{Absolute:} no re-excitation possible.\\
\emph{Relative:} AP only with stronger stimulus.\\
Reason: \(h \downarrow\) (\(I_{Na}\downarrow\)), \(n \uparrow\) (\(I_K\uparrow\)).\\

\textbf{Anode break excitation:}
AP after release from hyperpolarization due to
\(n \downarrow\), \(h \uparrow\), \(m \approx \text{normal}\)
\(\Rightarrow I_{Na} > I_K\).

\subsection*{Voltage Clamp}
