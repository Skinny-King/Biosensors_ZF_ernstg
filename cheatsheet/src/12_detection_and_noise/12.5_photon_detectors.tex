\subsection*{Photon Detectors}

\subsubsection*{Photomultiplier Tube (PMT)}

Single-pixel detector.
A photon releases an electron at the photocathode, which is multiplied at successive dynodes.
Gain is controlled by the acceleration voltage.

\textbf{Advantages:}
Low read noise.

\textbf{Disadvantages:}
Low quantum efficiency (10--15\%);
photon noise dominates;
no spatial resolution.

\subsubsection*{Charge-Coupled Device (CCD)}

Multi-pixel detector.
Photons generate electrons in pixels, forming charge clouds that are read out.

\textbf{Advantages:}
High quantum efficiency ($\sim$90\%);
parallel readout;
long exposure times possible.

\textbf{Disadvantages:}
Read noise can be significant;
fixed image size.

\textbf{Pixel binning:}
Charges of neighboring pixels are summed before readout.
Reduces read noise and increases speed, but lowers spatial resolution.

\textbf{Electron-multiplying CCD (EMCCD):}
CCD with an on-chip gain register.
Reduces read noise at the cost of increased photon noise.
