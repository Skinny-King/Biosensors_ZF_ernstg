\subsection*{Helical Sensor}
\includegraphics[width=0.9\linewidth]{src/0_images/helix_parallel_plate_model.png}\\
\textbf{Capacitance between two parallel wires:}

\begin{minipage}[c]{0.6\linewidth}
\centering
\fbox{$
C
=
\frac{\pi\,\varepsilon_{\text{tot}}\,l}
{\ln\!\left(
\frac{D}{2r}
+
\sqrt{\left(\frac{D}{2r}\right)^2 - 1}
\right)}
$}
\end{minipage}
\hfill
\begin{minipage}[c]{0.3\linewidth}
\centering
\includegraphics[width=\linewidth]{src/0_images/helix.png}
\end{minipage}

\[
\varepsilon_{\text{tot}}
=
\frac{\varepsilon_{\text{air}}\,\varepsilon_{\text{PDMS}}\,
(d+2t_0)}
{2t_0\,\varepsilon_{\text{air}} + d\,\varepsilon_{\text{PDMS}}},\quad
D
=
d(\varepsilon)
+
2t_0
+
2r_0
\]
\[
d(\varepsilon)
=
\frac{1}{\pi}
\sqrt{
l_{\text{turn}}^{\,2}
-
\left(\frac{1+\varepsilon}{n}\right)^2
}
-
2(t_0 + r_0)
\]
\[
l_{\text{turn}}
=
\sqrt{
\left(\frac{1}{n}\right)^2
+
\bigl(\pi\,(d_0 + 2(t_0 + r_0))\bigr)^2
}
\]

\vspace{4pt}

\begin{minipage}[t]{0.56\linewidth}
\tiny
$C$ : capacitance between two helical fibers $\left[\mathrm{F}\right]$ \\
$l_{\text{turn}}$ : length of one unwound helix turn $\left[\mathrm{m}\right]$ \\
$n$ : number of helix turns per unit length $\left[\mathrm{1/m}\right]$ \\
$\varepsilon_{\text{tot}}$ : effective permittivity (air + PDMS) $\left[\mathrm{F/m}\right]$ \\
$\varepsilon_{\text{air}}$ : permittivity of air $\left[\mathrm{F/m}\right]$ \\
$\varepsilon_{\text{PDMS}}$ : permittivity of PDMS $\left[\mathrm{F/m}\right]$ \\
$l$ : effective overlapping fiber length $\left[\mathrm{m}\right]$ \\
\end{minipage}
\begin{minipage}[t]{0.45\linewidth}
\tiny
$r_0$ : fiber core radius $\left[\mathrm{m}\right]$ \\
$t_0$ : PDMS coating thickness $\left[\mathrm{m}\right]$\\
$d(\varepsilon)$ : inter-fiber gap under strain $\left[\mathrm{m}\right]$ \\
$D$ : center-to-center fiber distance $\left[\mathrm{m}\right]$ \\
$d_0$ : initial helix diameter $\left[\mathrm{m}\right]$ \\
$\varepsilon$ : applied axial strain $\left[-\right]$
\end{minipage}
