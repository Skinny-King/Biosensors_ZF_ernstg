\subsection*{Airy Patterns}

A point source forms a diffraction-limited Airy pattern.
Smaller wavelength or larger numerical aperture (NA) reduce its size.
Images are formed by overlapping Airy patterns.\\
\textbf{Numerical aperture:}
\[
NA \uparrow \;\Rightarrow\; \text{better light collection and resolution}
\]
\begin{minipage}[t]{0.48\linewidth}
    \textbf{Resolution:} \\
    (Rayleigh criterion)
    \[
    R = \frac{0.61\,\lambda_{\mathrm{em}}}{NA}
    \]
\end{minipage}
\begin{minipage}[t]{0.48\linewidth}
    \textbf{Optimal pixel size:}\\
    (Nyquist)
    \[
    \mathrm{OPS} = \frac{M\, R}{2}
    \]
\end{minipage}

\textbf{Sampling:}\\
$p \le \mathrm{OPS}$: Nyquist satisfied, higher SNR near OPS (binning possible).\\
$p > \mathrm{OPS}$: increase $M$ or reduce pixel size.

\subsubsection*{Confocal Microscope}

A pinhole blocks out-of-focus light, improving axial resolution at the cost of SNR.

\begin{minipage}[t]{0.48\linewidth}
\vspace{0pt}
\textbf{Optimal pinhole size:}
\[
d_{\text{pinhole}} \approx \frac{0.61\,\lambda\, M}{NA}
\]
\end{minipage}
\hfill
\begin{minipage}[t]{0.48\linewidth}
\vspace{0pt}
\centering
\includegraphics[width=\linewidth]{src/0_images/confocal_microscopy.png}
\end{minipage}


\textbf{Spinning-disk:} parallel scanning, faster imaging.\\
\textbf{Two-photon:} excitation only at focus ($P\propto I^2$), deeper penetration, less photobleaching.
