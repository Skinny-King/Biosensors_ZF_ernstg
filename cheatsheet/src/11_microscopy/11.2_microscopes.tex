\subsection*{Microscopes}
\subsubsection*{Compound Microscope}

Combination of objective and eyepiece for high magnification.

\textbf{Infinity correction:}
Object in the front focal plane of the objective; parallel rays are focused by a tube lens.

\textbf{Magnification:}
\[
M = M_{\mathrm{obj}} \cdot M_{\mathrm{eye}} \cdot M_C
= \frac{160}{f_{\mathrm{obj}}} \cdot \frac{250}{f_{\mathrm{eye}}} \cdot M_C
\]

Tube length $=160\,\mathrm{mm}$, $M_C$ additional magnification.
\subsubsection*{Diascopic Fluorescence Microscope}

Sample is illuminated from below, between light source and objective.

\textbf{Limitation:}
Incomplete suppression of excitation light reduces contrast.

\textbf{Detected intensity:}
\[
I \propto \frac{NA_{\mathrm{obj}}^{2}}{M^{2}}, \quad
NA = n \sin\alpha
\]

\subsubsection*{Epi-Illumination Microscope}
Excitation and emission light share the same optical path via the objective.
A dichroic beam splitter reflects excitation light and transmits fluorescence emission,
resulting in high contrast.

\begin{minipage}[t]{0.5\linewidth}
\vspace{0pt}
\centering
\includegraphics[width=\linewidth]{src/0_images/epi_ill_microscope.png}
\end{minipage}
\hfill
\begin{minipage}[t]{0.48\linewidth}
\vspace{0pt}

\textbf{Field of view:}
\[
\mathrm{FOV}
= \frac{\text{sensor size}}{\text{magnification}}
\]
\[
= \frac{D_{\text{input}}}{M}
\]

\textbf{Detected intensity:}
\[
I \propto \frac{NA^{4}}{M^{2}}
\]

{\tiny
$\mathrm{FOV}$ : field of view \\
$D_{\text{input}}$ : sensor size (active dimension) $\left[\mathrm{m}\right]$ \\
$M$ : total magnification $[-]$ \\
$NA$ : numerical aperture of objective $[-]$ \\
$I$ : detected fluorescence intensity\\
$n$ = refractive index of the medium (air, water, oil)$[-]$ \\
$\alpha$ = half opening angle of objective $[\mathrm{rad}]$ \\
}
\end{minipage}
